\documentclass[labelsBySect]{seto}
\title{seto.cls Sample(a nyan.cls variant)}
\author{Neal Yan heheheh}
\date{May 2021}
\begin{document}
\maketitle
A demonstration of the features of seto.cls.
\section*{Package information}
There is an option called \texttt{labelsBySect}, which labels problems, theorems, etc by section;\\
Another option is called \texttt{noheader}, which removes the headers identifying the author, etc.
\toc 
\section{This is a section header}
\subsection{Subsection appearance}
The command for (sub)(sub)section has not changed, although its appearance has: \fakecmd{(sub)(sub)section}.
\subsubsection{Just in case!}
\section{Math commands}
A few math commands have been defined, such as \fakecmd{symsum} and \fakecmd{cycsum}, which appear as $\symsum$ and $\cycsum$, respectively\footnote{sorry dennis lol}.\\
Some other ones defined include \fakecmd{deg}(changed from `deg' to `$\deg$'), and \fakecmd{pow}(power of a point).
\subsection{Delimiters(\textbackslash left and \textbackslash right already included)}
\begin{itemize}
    \item\fakecmd{abs}; aboslutelllly values: $\abs{pqr}$;
    \item\fakecmd{norm}; norms [of vectors]: $\norm{\textbf{v}}$;
    \item\fakecmd{floor}, \fakecmd{ceil}; floors/ceilings: $\floor{abc}$, $\ceil{abc}$;
    \item\fakecmd{parenth}; parentheses because characters typed is still less: $\parenth{xyz}$;
    \item\fakecmd{sqbrack}; brackets, just in case?
    \item\fakecmd{braces}; braces because why not: $\braces{n/k}$;
\end{itemize}
\fakecmd{evalat[upper]\{lower\}}$\Rightarrow\evalat[\text{first argument}]{\text{second argument}}$, because that kind of is a delimiter? It's not \fakecmd{mid} that creates the bar, duh...
\subsection{Floors/ceilings}
Due to personal need\footnote{Ross quiz, anyone?}, I've defined floor brackets because ``\fakecmd{left}\fakecmd{lfloor} \dots\fakecmd{right}\fakecmd{rfloor}" was too tedious to repeat.
\subsection{Sets}
I also found typing \fakecmd{mathbb\{R\}} for $\R$ too tedious to repeat, so I shortened it to \fakecmd{R}. Similar commands exist for $\N,\Z,\Q$, and $\C$:
\begin{center}
\begin{tabular}{c|c|c|c|c|c}
\emph{Command}& \fakecmd{N} & \fakecmd{Z} & \fakecmd{Q} & \fakecmd{R} & \fakecmd{C}\\\hline
\emph{Appearance}& $\N$ & $\Z$ & $\Q$ & $\R$ & $\C$
\end{tabular}  
\end{center}
\section{Fonts/text formatting}
On this sty I changed up the fonts:
\begin{itemize}
    \item Title/section/quotes font is {\Alegreya Alegreya};
    \item Sans font used is {\lato Lato}.
    \item Body text font is EB Garamond.
\end{itemize}
\fakecmd{emph}'s appearance has changed as well, \emph{like so}.
For itemize I'll change up the item icons later, using TikZ probably.\\
\newpage
\quote{Quotes are called using `\fakecmd{quote\{text\}\{author\}}',\\
although this type looks more suitable at the top of a page.}{seto.sty}
\section{Fancy stuff}
\problem{The appearance of a problem/example. Who needs environments, thus these boxes are defined as commands instead of environments. This one is called by \fakecmd{problem/example[<label>]\{<text>\}}. When the first argument is omitted, the labeling parentheses will also disappear.}
\theorem{Theorems and definitions are this color. (\fakecmd{theorem/definition\{\}})}
\lemma{Lemmas/claims are this color. (\fakecmd{lemma/claim\{\}})}
\pro{I'm also too lazy to add a frame/box to enclose proofs, so whatever. (\fakecmd{pro\{\}})}
\remark{I used a [hopefully] better color theme(for anyone who cares, it's Big Sur). Oh, and remarks/notes are called by \fakecmd{remark/note\{...\}}.}
Oh, and to remove the labeling numbers, use a shortened version of the original command:
\begin{center}
\begin{tabular}{c|c|c|c|c}
    \emph{Original}& \fakecmd{problem} & \fakecmd{example}&\fakecmd{theorem}&\fakecmd{definition}\\\hline
    \emph{Shortened} & \fakecmd{prob} & \fakecmd{exmp}&\fakecmd{theo}&\fakecmd{defn}
\end{tabular}  
\end{center}
The other commands do not have abbreviated versions.\\
\emph{Table of Contents: }it finally worked!!!! Command is \fakecmd{toc}.
\section{Exercises}
\noindent
\exercise The bolded problem number and label is called by \fakecmd{exercise}. It takes an optional input as a label.
\exercise[sample label] Another exercise.
\begin{sol}
Solution works intuitvely. (This is an environment called \texttt{sol}.)
\end{sol}
\end{document}
