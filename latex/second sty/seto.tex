\documentclass{article}
\usepackage{seto}
\title{seto.sty Sample\\(a nyanya.sty variant)}
\author{Neal Yan heheheh}
\date{May 2021}

\begin{document}

\maketitle

\toc

\section{This is a section header}
\subsection{Subsection appearance}

The command for (sub)(sub)section has not changed, although its appearance has: \texttt{\textbackslash(sub)(sub)section}.

\subsubsection{Just in case you need these.}

There is a slight possibility of the necessity of this command, \texttt{\textbackslash subsubsection}, I guess.

\section{Math commands}

A few math commands have been defined, such as \texttt{\textbackslash sym} and \texttt{\textbackslash cyc}, which appear as $\sym$ and $\cyc$, respectively\footnote{Inspired by shootinglucky in his \emph{lucky.sty}}.

\subsection{Between bars}

After noticing the lack of absolute value bars and the evaluated-at thing, the following were defined: 

\texttt{\textbackslash abs\{...\}}$\Rightarrow\abs{\dots}$;

\texttt{\textbackslash evalat[upper number]\{lower number\}}$\Rightarrow\evalat[\text{first argument}]{\text{second argument}}$.

\subsection{Floors/ceilings}
Due to personal need\footnote{Ross quiz, anyone?}, I've defined floor brackets because ``\texttt{\textbackslash lfloor \dots \textbackslash  rfloor}" was too tedious to repeat.  Furthermore, the brackets can be of specified size. The basic commands are \texttt{\textbackslash floor} and \texttt{\textbackslash ceil}. They may be made larger by prefixing \texttt{big}, \texttt{bigg}, or \texttt{Bigg} before \texttt{floor\{...\}}/\texttt{ceil\{...\}}, which already include \texttt{\textbackslash left} and \texttt{\textbackslash right} and will change size as according to what is within the floor/ceil bars:
\begin{center}
\begin{tabular}{c|c|c|c|c}
    \emph{Command} & \texttt{\textbackslash floor} & \texttt{\textbackslash bigfloor} & \texttt{\textbackslash biggfloor} & \texttt{\textbackslash Biggfloor}\\\hline
    \emph{Appearance} & $\floor{\enskip}$ & $\bigfloor{\enskip}$ & $\biggfloor{\enskip}$ & $\Biggfloor{\enskip}$ \\
\end{tabular}  
\end{center}

\subsection{Sets}

I also found typing \texttt{\textbackslash mathbb\{R\}} for $\R$ too tedious to repeat, so I shortened it to \texttt{\textbackslash R}. Similar commands exist for $\N,\Z,\Q$, and $\C$:
\begin{center}
\begin{tabular}{c|c|c|c|c|c}
\emph{Command}& \texttt{\textbackslash N} & \texttt{\textbackslash Z} & \texttt{\textbackslash Q} & \texttt{\textbackslash R} & \texttt{\textbackslash C}\\\hline
\emph{Appearance}& $\N$ & $\Z$ & $\Q$ & $\R$ & $\C$
\end{tabular}  
\end{center}

\section{Fonts/text formatting}

On this sty I changed up the fonts.
To \emph{emphasize} a word, the command \texttt{\textbackslash emph} is still used.

As for the environments \texttt{enumerate}, \texttt{itemize}, and \texttt{enumitem}, I'm still too lazy to change those, lel.

Also, the command \texttt{\textbackslash solution}(which compiles as "\textit{Solution. }", because \texttt{\textbackslash textit\{Solution. \}} is also rather long.
\newpage
\quote{Quotes are called using `\textbackslash quote\{text\}\{author\}', although this type looks more suitable at the top of a page.}{seto.sty}
\section{Fancy stuff}
\problem{The appearance of a problem/example. It is called by the command 

\texttt{\textbackslash problem/example[optional argument for source]\{problem text\}}. When the first argument is omitted, it will display ``source unknown" in place of a source.}
\theorem{Theorems and definitions are this color. (\texttt{\textbackslash theorem/definition\{\}})}
\lemma{Lemmas/claims are this color. (\texttt{\textbackslash lemma/claim\{\}})}
\pro{I do not think proofs need their own box... at least not now. The command for proofs like this one is \texttt{\textbackslash pro\{proof text\}}.}

\remark{I used a [hopefully] better color theme(for anyone who cares, it's Big Sur). Oh, and remarks are called by \texttt{\textbackslash remark\{...\}}.}

Oh, and to remove the labeling numbers, use a shortened version of the original command:
\begin{center}
\begin{tabular}{c|c|c|c|c|c}
    \emph{Original}& \texttt{\textbackslash (sub)(sub)section} & \texttt{\textbackslash problem} & \texttt{\textbackslash example}&\texttt{\textbackslash theorem}&\texttt{definition}\\\hline
    \emph{Shortened}& \texttt{\textbackslash (sub)(sub)sctn} & \texttt{\textbackslash prob} & \texttt{\textbackslash exmp}&\texttt{\textbackslash theo}&\texttt{\textbackslash defn}
\end{tabular}  
\end{center}

The other commands do not have abbreviated versions.

\emph{Table of Contents: }it finally worked!!!! Command is \texttt{\textbackslash toc}.

\section{Exercises}
\noindent
\exercise The bolded problem number and label is called by \texttt{\textbackslash exercise}. (who needs inputs lol)
\exercise Another exercise. 

\remark{Still rather unrefined, the typewriter font is a lot bigger than the normal font. In any case, I only added a few new features, but some things were improved. I really couldn't resist changing the fonts and colors to ones of my liking.}

\inlinequote{Inline quotes( \textbackslash inlinequote\{\}\{\}) look like this.}{seto.sty}

I think that's about it! A bit more features by now\dots
\end{document}
