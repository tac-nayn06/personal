\documentclass[labelsBySect]{nyan}
\title{nyan.cls Sample}
\author{Neal Yan}
\date{April 2021\\ Presenting the most garish color theme to ever exist}
\begin{document}
\maketitle
\toc
\section{This is a section header}
\subsection{Subsection appearance}
The command for (sub)(sub)section has not changed, although its appearance has: \fakecmd{(sub)(sub)section}.
\subsubsection{Just in case you need these.}

There is a slight possibility of the necessity of this command, \fakecmd{subsubsection}, I guess.

\section{Math commands}

A few math commands have been defined, such as \fakecmd{sym} and \fakecmd{cyc}, which appear as $\symsum$ and $\cycsum$, respectively\footnote{hi dennis heheheheh}

\subsection{Between bars}

After noticing the lack of absolute value bars and the evaluated-at thing, the following were defined: 

\fakecmd{abs\{...\}}$\Rightarrow\abs{\dots}$;

\fakecmd{evalat[upper number]\{lower number\}}$\Rightarrow\evalat[\text{first argument}]{\text{second argument}}$.

\subsection{Floors/ceilings}
Due to personal need\footnote{Ross quiz, anyone?}, I've defined floor brackets because ``\fakecmd{lfloor \dots \textbackslash  rfloor}" was too tedious to repeat. 
\subsection{Sets}

I also found typing \fakecmd{mathbb\{R\}} for $\R$ too tedious to repeat, so I shortened it to \fakecmd{R}. Similar commands exist for $\N$,$\Z$,$\Q$, and $\C$:
\begin{center}
\begin{tabular}{c|c|c|c|c|c}
\emph{Command}& \fakecmd{N} & \fakecmd{Z} & \fakecmd{Q} & \fakecmd{R} & \fakecmd{C}\\\hline
\emph{Appearance}& $\N$ & $\Z$ & $\Q$ & $\R$ & $\C$
\end{tabular}  
\end{center}

\section{Fonts/text formatting}

Default fonts used, emph appearance changed: \emph{like so}

As for the environments \texttt{enumerate}, \texttt{itemize}, and \texttt{enumitem}, I'm too lazy to change those, lel.

Also, the command \fakecmd{solution}(which compiles as "\textit{Solution. }", because \fakecmd{textit\{Solution. \}} is also rather long.

\section{Fancy stuff}
\problem{The appearance of a problem/example. Command is \fakecmd{problem[]\{\}.}}
\theorem{Theorems and definitions are this color. (\fakecmd{theorem/definition\{\}})}
\lemma{Lemmas/claims are this color. (\fakecmd{lemma/claim\{\}})}
\pro{I do not think proofs need their own box... at least not now. The command for proofs like this one is \fakecmd{pro\{proof text\}}.}

\remark{The tasteless color theme is from Nyan Cat(excluding the rainbows), judge me. Oh, and remarks are called by \fakecmd{remark\{...\}}.}

Oh, and to remove the labeling numbers, use a shortened version of the original command:
\begin{center}
\begin{tabular}{c|c|c|c|c|c}
    \emph{Original}& \fakecmd{(sub)(sub)section} & \fakecmd{problem} & \fakecmd{example}&\fakecmd{theorem}&\texttt{definition}\\\hline
    \emph{Shortened}& \fakecmd{(sub)(sub)sctn} & \fakecmd{prob} & \fakecmd{exmp}&\fakecmd{theo}&\fakecmd{defn}
\end{tabular}  
\end{center}

The other commands do not have abbreviated versions.

\emph{Table of Contents: }it finally worked!!!! Command is \fakecmd{toc}.

\section{Exercises}
\noindent
\exercise The bolded problem number and label is called by \fakecmd{exercise}. (who needs inputs lol)
\exercise[random label] Another exercise. 
\begin{sol}[argument for label]
Solution environment, called sol.
\end{sol}
\\[4pt]
That's about all nyan.cls has to it!\\

\remark{This is my first .cls, and in later styles I've improved the color theme and changed up the fonts, see seto.cls.}
\end{document}
